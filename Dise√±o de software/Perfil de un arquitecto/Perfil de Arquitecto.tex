\documentclass[a4paper,12pt]{article}

\usepackage[top=1in, bottom=1in, left=1in, right=1in]{geometry}
\usepackage[spanish]{babel}
\usepackage[utf8]{inputenc}
\usepackage[T1]{fontenc}
\usepackage{lmodern}
\usepackage{setspace}
\setcounter{page}{3}

\begin{document}
\title{Perfil ideal de un arquitecto de software}
\date{}
\author{Instituto Tecnologico de Costa Rica, Escuela de Ingeniería en Computación\\
Curso de Diseño de Software, Profesora Ericka Solano Fernandez\\
Autor: Pedro Henrique Rodriguez de Oliveira (2013086585)}
\maketitle

\onehalfspacing
\begin{itemize}
  \item Ser capaz de abstraer muy bien todos las ideas que se planean realizar, ya que es importante para todos los miembros del equipo entender muy bien los aspectos estructurales del istema y que ellos puedan ejecutar mejor sus estrategias de desarrollo.
  \item Ser un buen comunicador de ideas, porque sería problemático tener un líder que no sepa como comunicarse con los miembros del equipo y que hayan desacuerdos más adelante.
  \item Saber muy bien de tecnología, porque es necesario saber cuáles herramientas son las requeridas para poder realizar de mejor manera una cierta labor. Por ejemplo, si se hace un programa de estadística, si se requieren hacer scripts, sería recomendable usar un lenguaje como R, ya que está más destinado a hacer labores de estadística.
  \item Ser también capaz de actuar como un administrador de proyectos, ya que estos tienen una labor muy similar y los conocimientos en ambas áreas se complementan fuertemente.
\end{itemize}

\begin{thebibliography}{50}
	\bibitem{Bredemeyer1} Bredemeyer, D. y Malan, R. (28 Abril 2005). “Software Architecture: Central Concerns, Key Decisions”. Extraído el 15 de febrero del 2016 de la página web http://www.bredemeyer.com/pdf\_files/ArchitectureDecisions.PDF
	\bibitem{Bredemeyer2} Bredemeyer, D. y Malan, R. (28 Abril 2005). “The Role of the Architect”. Extraído el 15 de febrero del 2016 de la página web http://www.bredemeyer.com/pdf\_files/role.pdf
\end{thebibliography}

\end{document}